\documentclass[letterpaper,12pt]{article}
\usepackage[utf8]{inputenc}
%\usepackage[top=1.25in, bottom=1.25in, left=1in, right=1in]{geometry}
% Adjusted margins to reduce the number of pages:
\usepackage[top=0.9in, bottom=0.9in, left=0.5in, right=0.5in]{geometry}
\usepackage{enumitem}
\usepackage{titlesec}
\usepackage{hyperref}

\setlength\parindent{0cm}
\setitemize{itemsep=-2pt}
\titleformat{\section}{\normalfont\Large\bfseries}{\thesection}{1em}{}[{\titlerule[0.8pt]}]

\begin{document}

\moveleft.5\hoffset\centerline{\Large\bf Benjamin E. Noland}
\smallskip
\moveleft.5\hoffset\centerline{609-851-1137}
\moveleft.5\hoffset\centerline{benjaminnoland93@gmail.com}
%\moveleft.5\hoffset\centerline{benjamin.noland@rutgers.edu}
\moveleft.5\hoffset\centerline{\url{http://bnoland.github.io/}}

\section*{Education}

\begin{itemize}
\item \textbf{MS in statistics, Rutgers University}, New Brunswick, NJ
  (with honors) \\
  September 2017--May 2019

\item \textbf{BA in mathematics, Rutgers University}, New Brunswick, NJ
  (with honors) \\
  September 2012--May 2016 \\
  Minor in physics \\
  School of Arts and Sciences Honors Program
\end{itemize}

\iffalse
\subsection*{Selection of coursework:}

\begin{itemize}
\item \textbf{Mathematics:} Calculus, linear algebra, ordinary
  differential equations, real analysis, complex variables,
  differential geometry, linear programming, abstract algebra,
  topology \textit{(taken at Rutgers University)}

\item \textbf{Physics:} Classical mechanics, electromagnetism,
  astrophysics \textit{(taken at Rutgers University)}

\item \textbf{Computer science:} Systems programming, data structures
  and algorithms \textit{(taken at Princeton University while in high
    school)}; Advanced Placement computer science \textit{(taken in
    high school)}

\item \textbf{Statistics:} Advanced Placement statistics
  \textit{(taken in high school)}, probability theory, regression
  analysis \textit{(taking at Rutgers University)}

\end{itemize}
\fi

\section*{Experience}

\textit{\textbf{Freelance statistician}} \\
Princeton, NJ \\
June, 2020--Present
\begin{itemize}
\item I currently offer data analysis and programming services through
  freelancer sites such as Guru and Upwork.
\end{itemize}

\textit{\textbf{Independent mathematics tutor}} \\
Princeton, NJ \\
February 2020--March 2020
\begin{itemize}
\item Tutored students from Princeton High School and Princeton
  University in mathematics.
\end{itemize}

\textit{\textbf{Part-time research assistant}} \\
\textbf{Rutgers Infrastructure Monitoring and Evaluation Group,
  Department of Civil and Environmental Engineering, Rutgers
  University},
New Brunswick, NJ \\
June 2019--November 2019
\begin{itemize}
\item Used R to process and manage data on bids made by contractors
  for construction jobs nationwide, using data from the Bid Express
  bidding service.
\item Assisted in research investigating bid price distributions and
  forecasting future bid prices. Included extensive exploratory
  analysis, including producing visualizations using ggplot, along
  with analysis of the time series of bid prices using standard R
  modeling tools.
\item Assisted in writing a report for the New Jersey Department of
  Transportation on bid price distributions for construction bids
  statewide. The report was written in RMarkdown, and the process
  automated using the drake R library.
\item \textbf{Skills used:} R, data processing, statistics
\end{itemize}

\textit{\textbf{Part-time research assistant}} \\
\textbf{Alan M. Voorhees Transportation Center, Bloustein School of
  Planning and Public Policy, Rutgers University},
New Brunswick, NJ \\
June 2018--July 2018
\begin{itemize}
\item Used R and the Elsevier Scopus API to scrape roughly two-decades
  worth of abstracts from articles published in transportation-related
  journals. The goal was to determine whether writing quality in these
  journals has degraded or improved with time.
\item The scraped data was cleaned using R and fed into the Coh-Metrix
  system, developed at the University of Memphis to analyze linguistic
  cohesion among a collection of text corpora.
\item No conclusive results from this brief study, but open to further
  investigation.
\item \textbf{Skills used:} R, data processing
\end{itemize}

\textit{\textbf{Part-time research assistant}} \\
\textbf{School of Management and Labor Relations, Rutgers University},
New Brunswick, NJ \\
May 2018--May 2019
\begin{itemize}
\item Designed and implemented a web application using R and Shiny to
  explore the unionization trends of registered nurses in the United
  States using Current Population Survey (CPS) data.
\item Included extensive preprocessing of the raw data using R.
\item The application allows the user to select, aggregate, and
  visualize the data to explore union membership and union contract
  coverage rates. The application was built using the Shiny framework,
  with visualizations done using ggplot.
\item Wrote extensive documentation for the tool, formatted using
  \LaTeX.
\item The tool is currently in use by the School of Management and
  Labor Relations at Rutgers and other interested parties to analyze
  unionization trends among registered nurses in the United States.
\item The application is currently available at:
  \begin{center}
    %\url{https://bnoland.shinyapps.io/nurses_web_tool/}
    \url{https://smlr.rutgers.edu/content/nurse-unionization-data-tool}
  \end{center}
  The latest version of the code can be found at:
  \begin{center}
    \url{https://github.com/bnoland/nurses_web_tool}
  \end{center}
\item \textbf{Skills used:} R, Shiny, data processing
\end{itemize}

\textit{\textbf{Independent mathematics tutor}} \\
\textbf{Mercer County Community College},
West Windsor, NJ \\
May 2017--July 2017
\begin{itemize}
\item Established an independent tutoring service.
\item Tutored students from Mercer County Community College in
  mathematics (mainly calculus and precalculus).
\end{itemize}

\textit{\textbf{Programming intern}} \\
\textbf{Alan M. Voorhees Transportation Center, Bloustein School of
  Planning and Public Policy, Rutgers University},
New Brunswick, NJ \\
June 2016--November 2016
\begin{itemize}
\item Formulated heuristic methods for detecting possible groups of
  riders in Citi Bike trip data.
\item Designed and implemented R scripts to process the raw data and
  implement these methods. The latest versions of the scripts can be
  found at:
  \begin{center}
    \url{https://github.com/bnoland/citibike}
  \end{center}

\item Implemented a website for visualizing the results of this study,
  using Google Fusion Tables to store the processed data and the
  Google Maps API to visualize it. The source can be found at:
  \begin{center}
    \url{https://github.com/bnoland/citibike-map}
  \end{center}

\item \textbf{Skills used:} R, data processing, HTML, CSS, JavaScript,
  SQL
\end{itemize}

\textit{\textbf{Programming intern}} \\
\textbf{Vertices, LLC}, New Brunswick, NJ \\
May 2015--August 2015
\begin{itemize}
\item Worked on Mappler, an online geographic information system (GIS)
  tool. Designed and implemented a feature that allows users to upload
  images, extracts GPS data from the images, and adds them to the map
  database. Used PHP for the backend code, and JavaScript for the
  frontend.
\item Implemented a daemon in Python for extracting images and
  associated GPS data from email accounts and adding them to a map
  database.
\item \textbf{Skills used:} HTML, CSS, JavaScript, PHP, Python, SQL
\end{itemize}

\textit{\textbf{Programming intern}} \\
\textbf{Alan M. Voorhees Transportation Center, Bloustein School of
  Planning and Public Policy, Rutgers University},
New Brunswick, NJ \\
July 2014--August 2014
\begin{itemize}
\item Designed and implemented a website that maps crashes involving
  vehicles and pedestrians (including bicyclists) using data provided
  by the New Jersey Department of Transportation.
\item The site allowed the user to submit search queries to filter the
  data. Google Fusion Tables was used to store the data, and
  visualization was done using the Google Maps API.
\item \textbf{Skills used:} HTML, CSS, JavaScript, SQL
\end{itemize}

\section*{Extracurricular Activities}

\textit{\textbf{Head of Computer Club}} \\
\textbf{Princeton High School},
Princeton, NJ \\
September 2009--February 2012
\begin{itemize}
\item Worked with club members towards developing a robot that could
  navigate a maze.
\item Organized fundraising for the club.
\item Taught other students the basics of programming.
\end{itemize}

% TODO: Keep additional experience section?  \iffalse
\iffalse
\section*{Additional Experience}

\textit{\textbf{Tutoring (informal)}} \\
\textbf{Rutgers University}, New Brunswick, NJ, September 2012-May
2016
\begin{itemize}
\item Provided informal tutoring in programming and mathematics to
  students at Rutgers University.
\end{itemize}

\textit{\textbf{Head of Computer Club}} \\
\textbf{Princeton High School}, Princeton, NJ, September 2009-February
2012
\begin{itemize}
\item Worked with club members towards developing a robot that could
  navigate a maze.
\item Organized fundraising for the club.
\item Taught other students the basics of programming.
\end{itemize}

\textit{\textbf{Video game development program}} \\
\textbf{Rensselaer Polytechnic Institute}, Troy, NY, July 2011
\begin{itemize}
\item Learned the basics of video game development.
\item Developed a small game in a team environment using Python and
  Pygame.
\end{itemize}

\textit{\textbf{Taught robotics to elementary school students}} \\
\textbf{Riverside Elementary School}, Princeton, NJ, January
2011-March 2011
\begin{itemize}
\item Used Lego to teach the elements of robotics to elementary school
  students.
\end{itemize}

\fi

\section*{Skills}

\subsection*{Statistics and data analysis skills:}

\begin{itemize}
\item Knowledge of statistical theory (classical and some Bayesian).
\item Knowledge of a variety of modeling techniques (for both
  inference and prediction).
\begin{itemize}
\item Ordinary least squares regression
\item ANOVA
\item Time series analysis
\item Multivariate analysis
\item Generalized linear models (e.g., logistic and Poisson
  regression)
\item Penalized linear regression methods (ridge regression and the
  LASSO)
\item Classification methods (e.g., logistic regression, LDA, QDA,
  KNN, SVMs)
\item Bootstrapping methods
\item Unsupervised techniques (e.g., clustering methods and PCA)
\item Deep learning and neural networks
\item Others (e.g., spline methods and tree-based methods)
\end{itemize}
\item Proficient in data processing using R (including data wrangling,
  modeling, and visualization), as well as building web applications
  with Shiny.
\item Experience with Python data analysis tools (NumPy, pandas,
  scikit-learn, etc.).
\item Experience using Keras to build deep learning models.
\item Some experience with Stata.
\end{itemize}

\subsection*{Computer skills:}

\begin{itemize}
\item \textbf{Proficient with:} R, Python, C, Java, \LaTeX, Unix,
  Windows, Git, Microsoft Office (and similar tools)
\item \textbf{Experience with:} MATLAB, Stata, JavaScript, HTML, CSS,
  PHP, MySQL, x86 assembly language
\item \textbf{GitHub account:} \url{https://github.com/bnoland}
\end{itemize}

\section*{Honors}
\begin{itemize}
\item \textbf{2014 Rutgers Academic Excellence Award}, April 2014
\item \textbf{Princeton High School Computer Science Award}, June 2012
\end{itemize}

\end{document}
