\documentclass[letterpaper,12pt]{article}
\usepackage[utf8]{inputenc}
%\usepackage[top=1.25in, bottom=1.25in, left=1in, right=1in]{geometry}
% Adjusted margins to reduce the number of pages:
\usepackage[top=0.9in, bottom=0.9in, left=0.5in, right=0.5in]{geometry}
\usepackage{enumitem}
\usepackage{titlesec}
\usepackage{hyperref}

\setlength\parindent{0cm}
\setitemize{itemsep=-2pt}
\titleformat{\section}{\normalfont\Large\bfseries}{\thesection}{1em}{}[{\titlerule[0.8pt]}]

\begin{document}

\moveleft.5\hoffset\centerline{\Large\bf Benjamin E. Noland}
\smallskip
\moveleft.5\hoffset\centerline{609-851-1137}
%\moveleft.5\hoffset\centerline{benjaminnoland93@gmail.com}
\moveleft.5\hoffset\centerline{benjamin.noland@rutgers.edu}
\moveleft.5\hoffset\centerline{\url{http://bnoland.github.io/}}

\section*{Education}

\begin{itemize}
\item
\textbf{Rutgers University}, New Brunswick, NJ, September 2017--May 2019 (expected) \\
MS in statistics

\item
\textbf{Rutgers University}, New Brunswick, NJ, September 2012--May 2016 \\
BA in mathematics, minor in physics (\textit{cum laude}) \\
School of Arts and Sciences Honors Program
\end{itemize}
\iffalse
\subsection*{Selection of coursework:}
\begin{itemize}
\item
\textbf{Mathematics:} Calculus, linear algebra, ordinary differential equations, real analysis, 
complex variables, differential geometry, linear programming, abstract algebra, topology 
\textit{(taken at Rutgers University)}

\item
\textbf{Physics:}
Classical mechanics, electromagnetism, astrophysics \textit{(taken at Rutgers University)}

\item
\textbf{Computer science:}
Systems programming, data structures and algorithms \textit{(taken at Princeton University while in 
high school)}; Advanced Placement computer science \textit{(taken in high school)}

\item
\textbf{Statistics:} Advanced Placement statistics \textit{(taken in high school)}, probability 
theory, regression analysis \textit{(taking at Rutgers University)}

\end{itemize}
\fi
\section*{Skills}

\subsection*{Statistics and data analysis skills:}
\begin{itemize}
\item
Knowledge of statistical theory (classical and some Bayesian).
\item
Knowledge of a variety of modeling techniques (for both inference and prediction).
\begin{itemize}
\item
Ordinary linear regression
\item
Generalized linear models (e.g., logistic and Poisson regression).
\item
Penalized linear regression methods (ridge regression and the LASSO).
\item
Classification methods (e.g., logistic regression, LDA, QDA, KNN, SVMs).
\item
Bootstrapping methods.
\item
Unsupervised techniques (e.g., clustering methods and PCA).
\item
Deep learning and neural networks.
\item
Others (e.g., spline methods and tree-based methods).
\end{itemize}
\item
Proficient in data processing using R (including data wrangling, modeling, and visualization), as 
well as building web applications with Shiny.
\item
Some experience with Stata, as well as some with SciPy/Jupyter.
\end{itemize}

\subsection*{Computer skills:}
\begin{itemize}
\item
\textbf{Proficient with:} R, Python, C, Java, \LaTeX, Unix, Windows, Git, Microsoft Office (and 
similar tools)
\item
\textbf{Experience with:} MATLAB, Stata, SciPy/Jupyter, JavaScript (including JQuery), HTML, CSS, PHP, 
MySQL, x86 assembly language
\item
\textbf{GitHub account:} \url{https://github.com/bnoland}
\end{itemize}

\section*{Experience}
\textit{\textbf{Part-time research assistant}} \\
\textbf{School of Management and Labor Relations, Rutgers University}, New Brunswick, NJ, May 
2018--September 2018
\begin{itemize}
\item
Designed and implemented a web application to explore the unionization trends of registered 
nurses in the United States using Current Population Survey (CPS) data.
\item
The application allows the user to select, aggregate, and visualize the data to explore union 
membership and union contract coverage rates.
\item
The application was built using R and Shiny. It is currently available at:
\begin{center}
\url{https://bnoland.shinyapps.io/nurses_web_tool/}
\end{center}
The latest version of the code may be found at:
\begin{center}
\url{https://github.com/bnoland/nurses_web_tool}
\end{center}
\item
Wrote extensive documentation for the tool.
\end{itemize}

\textit{\textbf{Programming intern}} \\
\textbf{Voorhees Transportation Center, Rutgers University}, New Brunswick, NJ, June 2016--November 
2016
\begin{itemize}
\item
Designed and implemented R scripts to detect possible groups of riders in Citi Bike trip data. The 
latest versions of the scripts may be found at:
\begin{center}
\url{https://github.com/bnoland/citibike}
\end{center}

\item
Implemented a website for visualizing the results of this study. The latest version may be found at:
\begin{center}
\url{https://bnoland.github.io/citibike-map/}
\end{center}

\end{itemize}

\textit{\textbf{Programming intern}} \\
\textbf{Vertices, LLC}, New Brunswick, NJ, May 2015--August 2015
\begin{itemize}
\item
Worked on Mappler, an online geographic information system (GIS) tool. Designed and implemented a 
feature that allows users to upload images, extracts GPS data from the images, and adds them to the 
map database.
\item
Implemented a daemon for extracting images and associated GPS data from email accounts and adding 
them to a map database.
\end{itemize}

\textit{\textbf{Programming intern}} \\
\textbf{Voorhees Transportation Center, Rutgers University}, New Brunswick, NJ, July 2014--August 
2014
\begin{itemize}
\item
Designed and implemented a website that maps crashes involving vehicles and pedestrians (including 
bicyclists) using data provided by the New Jersey Department of Transportation.
\item
The site allows the user to submit search queries to filter the data. The site can be found at:
\begin{center}
\url{http://pppolicy.rutgers.edu/vtcdata/pedestrian/pedmap.html}
\end{center}
\end{itemize}

\section*{Extracurricular Activities}

\textit{\textbf{Head of Computer Club}} \\
\textbf{Princeton High School}, Princeton, NJ, September 2009--February 2012
\begin{itemize}
\item
Worked with club members towards developing a robot that could navigate a maze.
\item
Organized fundraising for the club.
\item
Taught other students the basics of programming.
\end{itemize}

% TODO: Keep additional experience section?
\iffalse

\section*{Additional Experience}

\textit{\textbf{Tutoring (informal)}} \\
\textbf{Rutgers University}, New Brunswick, NJ, September 2012-May 2016
\begin{itemize}
\item
Provided informal tutoring in programming and mathematics to students at Rutgers University.
\end{itemize}

\textit{\textbf{Head of Computer Club}} \\
\textbf{Princeton High School}, Princeton, NJ, September 2009-February 2012
\begin{itemize}
\item
Worked with club members towards developing a robot that could navigate a maze.
\item
Organized fundraising for the club.
\item
Taught other students the basics of programming.
\end{itemize}

\textit{\textbf{Video game development program}} \\
\textbf{Rensselaer Polytechnic Institute}, Troy, NY, July 2011
\begin{itemize}
\item
Learned the basics of video game development.
\item
Developed a small game in a team environment using Python and Pygame.
\end{itemize}

\textit{\textbf{Taught robotics to elementary school students}} \\
\textbf{Riverside Elementary School}, Princeton, NJ, January 2011-March 2011
\begin{itemize}
\item
Used Lego to teach the elements of robotics to elementary school students.
\end{itemize}

\fi

\section*{Honors}
\begin{itemize}
\item
\textbf{2014 Rutgers Academic Excellence Award}, April 2014
\item
\textbf{Princeton High School Computer Science Award}, June 2012
\end{itemize}

\end{document}
